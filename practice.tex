\documentclass[t]{beamer}
\usecolortheme{crane} 
\useinnertheme{circles}
\useinnertheme{rectangles}

% Работа с русским языком
\usepackage{cmap}					% поиск в PDF
\usepackage{mathtext} 				% русские буквы в формулах
\usepackage[T2A]{fontenc}			% кодировка
\usepackage[utf8]{inputenc}			% кодировка исходного текста
\usepackage[english,russian]{babel}	% локализация и переносы

% Дополнительная работа с математикой
\usepackage{amsmath,amsfonts,amssymb,amsthm,mathtools} % AMS
\usepackage{icomma} % "Умная" запятая: $0,2$ --- число, $0, 2$ --- перечисление

% Свои команды
\DeclareMathOperator{\sgn}{\mathop{sgn}}

% Перенос знаков в формулах (по Львовскому)
\newcommand*{\hm}[1]{#1\nobreak\discretionary{}
{\hbox{$\mathsurround=0pt #1$}}{}}

% Другие пакеты
\usepackage{lastpage} % Узнать, сколько всего страниц в документе.
\usepackage{soul} % Модификаторы начертания
\usepackage{csquotes} % Еще инструменты для ссылок
%\usepackage[style=authoryear,maxcitenames=2,backend=biber,sorting=nty]{biblatex}
\usepackage{multicol} % Несколько колонок
\usepackage{listings}
\usepackage{relsize}
\usepackage{hyperref}
\hypersetup{urlcolor=blue}

\title{broken oracle (HackTM CTF) 2023}

\author{Бакиновский Михаил}
\date{1 июля 2023 г.}
\institute[БФУ им. Иммануила Канта]{БФУ им. Иммануила Канта\\ ОНК «Институт высоких технологий» \\ Компьютерная безопасность, 3 курс}

\begin{document}

\frame[plain]{\titlepage}	% Титульный слайд

\section{Условие задачи broken oracle 2023}
\subsection{Описание с CryptoHack}
\begin{frame}
	\frametitle{\insertsection}
	\framesubtitle{\insertsubsection}
        Описание к заданию: \slshape"Я заново реализовал криптосистему, но иногда она ведет себя странно. Но я не думаю, что это имеет значение"\upshape\newline
        Так же к задаче прилагаeтся файл \slshape server.py.  \upshape \newline server.py содержит код на языке Python. В нём дана реализация криптографического алгоритма Рабина. При запуске кода нам предлагается ввести некоторое значение (зашифрованное сообщение) в формате \texttt{(r,s,t)}, на вывод мы также получим \texttt{(r,s,t)} - что является шифровкой расшифровки нашего изначального сообщения\newline
        Понятно, что нам нужно каким-то образом извлечь флаг или используя уязвимости криптосистемы или поняв закономерность шифрования.

\end{frame}

\section{Анализ задачи broken oracle 2023}
\subsection{Предварительный анализ}
\begin{frame}
	\frametitle{\insertsection}
	\framesubtitle{\insertsubsection}
        Важное уточнение: автор задания говорит, что это его видение следующей статьи:
        \begin{center} \href{https://www.cs.umd.edu/~gasarch/TOPICS/miscrypto/rabinwithrecip.pdf}{\textit{\textcolor{blue}{\fontsize{7pt}{1}\selectfont PUBLIC KEY CRYPTOSYSTEM USING A RECIPROCAL NUMBER}} }
        \end{center}
        Для начала проведем анализ криптосистемы.
        Нам дан оракул (консоль), в ответах которого иногда возникает ошибка. А также:\newline
        ${(n,c)}$-открытый ключ, состоящий из модуля ${n}$ и ${с}$, что выбирается случайным образом \newline
        Упрощая, нам необходимо найти приватный ключ ${(p,q)}$, после чего вычислить c и использовать эти значения для расшифровки флага.

\end{frame}

\section{Анализ задачи broken oracle 2023}
\subsection{Предварительный анализ}
\begin{frame}
	\frametitle{\insertsection}
	\framesubtitle{\insertsubsection}
    \begin{itemize}
        \item Атаки факторизацией не будут успешны в угоду того, что, технически, нам доступен только оракул, а число n - взято длиной 2048 бит, что делает факторизацию нерациональной по времени. Метод Ферма, к примеру, не сработает, ведь тут p и q - простые числа, взятые случайным образом, и это не означает того, что они будут "близки" друг другу.\newline
        \item Использовались ещё различные атаки: такие, как атака повторным шифрованием, встреча посередине и поиск квадратных корней по модулю n. Но они не увенчались успехом (за рациональное время).
    \end{itemize}
\end{frame}

\section{Анализ задачи broken oracle 2023}
\subsection{Анализ программы и вывода}
\begin{frame}
	\frametitle{\insertsection}
	\framesubtitle{\insertsubsection}
    После проведения серии запросов к оракулу было обнаружено следущее:
    \begin{itemize}
        \item Иногда при вводе значений r,s,t консоль выдаёт в ответ те же самые r,s,t, причём, если для такого r менять s и t - консольный вывод не изменится относительно r. Что, с точки зрения кода, должно выполняться всегда.
        \item Для произвольного шифра r,s,t происходит следующее: $(r_1,1,1)=(r_1,-1,0)$ и $(r_2,-1,1)=(r_2,1,0)$. Это говорит о том, что вместо 4 сообщений в расшифровке мы получаем лишь 2. На этом же основаны упрощения реализации криптосистемы и условия "assert len()==2"
    \end{itemize}
\end{frame}

\section{Анализ задачи broken oracle 2023}
\subsection{Анализ программы. Наблюдения}
\begin{frame}
	\frametitle{\insertsection}
	\framesubtitle{\insertsubsection}
    \begin{itemize}
        \item из 1000*k попыток 25\% - ситуация, когда $enc(r,s=1,t=1)=enc\_dnc\_enc(r=r,s=1,t=1)$, возникает ~в 25\% случаев для r в диапазоне от 1 до 1000*k и assertion error возникает в ~6\% случаев.
        \item Если символ лежандра для (r,n)=-1 то зашифрованное сообщение будет иметь единственный вариант расшифровки. Это связано с тем, что r будет являться квадратичным невычетом, а, значит, не будет иметь квадратных корней по модулю n, следовательно расшифровка будетт единственная и равна она будет зашифровке.
    \end{itemize}
\end{frame}

\section{Анализ задачи broken oracle 2023}
\subsection{Теоретические рассчёты}
\begin{frame}[t] % [t], [c], или [b] --- вертикальное выравнивание на слайде (верх, центр, низ)
	\frametitle{\insertsection}
	\framesubtitle{\insertsubsection}
            Если сравнивать реализацию криптосистемы Рабина в задаче с тем, как она представлена в открытом доступе, можно заметить ошибку в решении систем уравнений (solve\_squad):
        \begin{center}
            $x^2+rx+c=0\ mod\ p$
        \end{center}
            не имеет решений, когда
        \begin{center}
            $\left( \frac{r^2/4-c}{p}\right)=-1$
        \end{center}
            потому что
        \begin{center}
        $x^2+rx+c=0<=>(x+r/2)^2=r^2/4-c$
        \end{center}
\end{frame}

\section{Решение задачи broken oracle 2023}
\subsection{Получение p и q}
\begin{frame}
	\frametitle{\insertsection}
	\framesubtitle{\insertsubsection}
	   В функции decrypt возможен случай, когда число $r^2/4-c$ является квадратичным вычетом по модулю p, но не по модулю q.\newline Пусть зашифрованное сообщение r было расшифровано и результат равен r'. В этом случае $r=r'$ в GF(p) пока $r \neq r'$ в GF(q). Это означает, что $r-r'$ является квадратом некоторого числа по модулю p, но не по модулю q.\newline Поэтому, если обработать несколько $r'_i=Enc(Dec((r_i))$ - будет возможно восстановить p с помощью $НОД(r'_i-r_i,r'_j-r_j)$. Это также справедливо и для восстановления q. 
\end{frame}

\section{Решение задачи broken oracle 2023}
\subsection{Теоретические рассчёты}
\begin{frame}
	\frametitle{\insertsection}
	\framesubtitle{\insertsubsection}
    В нашей криптосистеме роль открытого ключа играет с. Однако, конкретно в данной задаче с не 
    доступно (так как это CTF). \newline
    Пусть $a_1,a_2=solve_{quad}(r,c,p)$ и $b_1,b_2=solve_{quad}(r,c,q)$, где r удовлетворяет уравнению \newline
    \begin{center}
    $\left( \frac{r^2/4-c}{p}\right)=\left( \frac{r^2/4-c}{q}\right)=-1$
    \end{center}
\end{frame}

\section{Решение задачи broken oracle 2023}
\subsection{Теоретические рассчёты}
\begin{frame}
	\frametitle{\insertsection}
	\framesubtitle{\insertsubsection}
    Если обработать $r_1,r_2$ как $r_1=Enc(m_1),r_2=Enc(m_2)$, где $m_1=a_1\ mod\ p,\ m_1=b_1\ mod\ q,\ m_2=a_2\ mod\ p,\ m_2=b_2\ mod\ q$ путём изменения параметров s и t с тем же самым r. Как только $m_2=r-m_1\ mod\ n$ следующие уравнения:
    \begin{center}
        $m^2_1-r_1m_1+c=0\ mod\ n$\newline
        $(r-m_1)^2-r_1(r-m_1)+c=0\ mod\ n$
    \end{center}
     Можно решить достаточно просто:
     \begin{center}
        $m_1=\frac{r_2r-r^2}{r_1-2r+r_2}\ mod\ n$\newline
        $c=r_1m_1-m^2_1\ mod\ n$ \newline
    \end{center}
    Таким образом, возможно узнать $c,m$ с помощью $r,r_1,r_2$\newline
    Также можно восстановить публичный и секретный ключ, чтобы расшифровать флаг.
\end{frame}
%----------------------------------------------------%
\section{Решение задачи broken oracle 2023}
\subsection{Код для решения данной задачи}
\begin{frame}[fragile]
\frametitle{\insertsection} 
\framesubtitle{\insertsubsection}
\footnotesize
\smaller
Функция, имитирующая запросы к оракулу.
\begin{lstlisting}[language=Python]
    def oracle(r,s,t,h=1):
        RST=encrypt(decrypt(Enc(r=r,s=s,t=t),pub,priv),pub)
        if h!=1: return RST
        return RST.r, RST.s, RST.t
\end{lstlisting}
Часть кода, оставшаяся от изначальной версии оракула.
\begin{lstlisting}
    pbits = 1024
    pub, priv = genkey(pbits)
\end{lstlisting}
Также уточню, что код изначальной задачи перенесн без изменений и дополнен.
Приводить его в презентации не рационально в связи с тем, что код достаточно обширен.
Потому далее приведу уникальную его часть, что дописана для решения задачи. Эта уникальная часть - алгоритм, описывающий (осуществляющий) ту идею, о которой говорилось ранее.
\end{frame}

\section{Решение задачи broken oracle 2023}
\subsection{Код для решения данной задачи 1 стр}
\begin{frame}[fragile]
\frametitle{\insertsection} 
\framesubtitle{\insertsubsection}
\footnotesize
\smaller
\begin{lstlisting}
        res = []                               
    for i in range(1, 21):
        rst = oracle(i, 1, 1)
        if rst[0] is None: continue
        res.append(rst[0] - i)
    factors = set()
    for i in range(len(res)):                
        if res[i] == 0: continue
        for j in range(i + 1, len(res)):
            if res[j] == 0: continue
            tmp = gcd(res[i], res[j])
            if tmp > 2**100:
                for pi in primerange(1000):
                    while True:
                        if tmp % pi == 0:
                            tmp //= pi
                        else: break
                factors.add(tmp)
\end{lstlisting}	
\end{frame}

\section{Решение задачи broken oracle 2023}
\subsection{Код для решения данной задачи 2 стр}
\begin{frame}[fragile]
\frametitle{\insertsection} 
\framesubtitle{\insertsubsection}
\footnotesize
\smaller
\begin{lstlisting}
    assert len(factors) == 2
    p = int(factors.pop())
    q = int(factors.pop())
    n = p * q
    print("Recover p, q, n:\np = %d \nq = %d \nn = %d"%(p,q,n))
    r = None
    for i in range(100):
        rst = oracle(i, 1, 1)
        if rst[0] is None: continue
        if gcd(rst[0] - i, n) == 1:
            r = i
            break 
\end{lstlisting}	
\end{frame}

\section{Решение задачи broken oracle 2023}
\subsection{Код для решения данной задачи 3 стр}
\begin{frame}[fragile]
\frametitle{\insertsection} 
\framesubtitle{\insertsubsection}
\footnotesize
\smaller
\begin{lstlisting}[language=Python]
    assert r is not None
    rs = []
    for s in [-1, 1]:
        for t in [0, 1]:
            rs.append(oracle(r, s, t)[0])
    for i in range(4):
        for j in range(i + 1, 4):
            r1 = rs[i]
            r2 = rs[j]
            try:
                m1 = (r2 * r - r ** 2) * pow(r1 - 2 * r + r2, -1, n) % n
                c = (r1 * m1 - m1 ** 2) % n
                print(long_to_bytes(decrypt(enc_flag, Pubkey(n=n, c=c), Privkey(p=p, q=q)))[0:22].decode())
            except Exception as e:
                print(e)
                continue
\end{lstlisting}	
\end{frame}

\section{Используемая литература}

\begin{frame}
    \frametitle{\insertsection}
    \begin{thebibliography}{1}
    \bibitem{2}Menezes, A., van Oorschot, P., and Vanstone, S. (1996). "Handbook of Applied Cryptography". CRC Press.\newline
    \bibitem{2}Stinson, D. R. (2006). "Cryptography: Theory and Practice". CRC Press. \newline 
    \bibitem{2}Shoup, V. (2009). "A Course in Number Theory and Cryptography". Graduate Texts in Mathematics, Vol. 114. Springer.
    \bibitem{2}Schneier, B. (1996). "Applied Cryptography: Protocols, Algorithms, and Source Code in C". John Wiley \& Sons.\newline
    \bibitem{2}Boneh, D. (2013). "Introduction to Cryptography". Online Course, Coursera.
    \end{thebibliography}
\end{frame}
\end{document}
